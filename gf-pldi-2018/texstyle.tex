%% =============================================================================
%% BEGIN custom tex styles

\usepackage{listings}
\usepackage{amsmath}
\usepackage{amssymb}
\usepackage{mathpartir}

% Override Scribble's default SecRef to numeric only
\renewcommand{\SecRef}[2]{#1}

\renewcommand{\vee}{\mbox{or}}

% For bib style
\newcommand{\Thyperref}[2]{\hyperref[#2]{#1}}

\let\captionwidth\relax

%% -----------------------------------------------------------------------------
%% common PL paper formatting

\newcommand{\subt}{\mathrel{<:}}
\newcommand{\subteq}{\mathrel{\leqslant:}}

%% -----------------------------------------------------------------------------
%% formatting Python code

% Default fixed font does not support bold face
\DeclareFixedFont{\ttb}{T1}{txtt}{bx}{n}{8} % for bold
\DeclareFixedFont{\ttm}{T1}{txtt}{m}{n}{8}  % for normal

% Custom colors
\definecolor{deepblue}{rgb}{0,0,0.5}
\definecolor{deepred}{rgb}{0.6,0,0}
\definecolor{deepgreen}{rgb}{0,0.5,0}

% Python style for highlighting
\newcommand\pythonstyle{\lstset{
language=Python,
basicstyle=\ttm,
commentstyle=\fontfamily{tt},
otherkeywords={self},             % Add keywords here
keywordstyle=\ttb\color{deepblue},
emph={MyClass,__init__},          % Custom highlighting
emphstyle=\ttb\color{deepred},    % Custom highlighting style
stringstyle=\color{deepgreen},
frame=none,                         % Any extra options here
showstringspaces=false            % 
}}

% Python environment
\lstnewenvironment{python}[1][]
{
\pythonstyle
\lstset{#1}
}
{}

% Python for external files
\newcommand\pythonexternal[1]{{
\pythonstyle
\lstinputlisting{#1}}}

% Python for inline
\newcommand\pythoninline[1]{{\pythonstyle\lstinline!#1!}}

%% -----------------------------------------------------------------------------
%% enable page numbers

\settopmatter{printfolios=true}

%% -----------------------------------------------------------------------------

\newcommand{\compilesto}{\rightsquigarrow}
\newcommand{\context}{\mathcal{C}}
\newcommand{\tagof}[1]{\lfloor #1 \rfloor}
\newcommand{\config}[1]{\langle #1 \rangle}
\newcommand{\stepsto}{\rightarrow^*}
\newcommand{\blameset}{\mathcal{B}}
\newcommand{\blame}[1]{\textsc{Blame}(#1)}
\newcommand{\labels}{\mathcal{L}}
\newcommand{\dyncheck}[2]{#1 \Downarrow \config{#2, l , \respath}}
\newcommand{\fresh}[1]{\emph{fresh}(#1)}
\newcommand{\respath}{\textsc{Res}}

\newcommand{\integers}{\mathbb{Z}}

\newcommand{\assign}{:\!=}

\newcommand{\tint}{\mathsf{Int}}
\newcommand{\tbox}{\mathsf{Box}}
\newcommand{\tlist}{\mathsf{List}}
\newcommand{\tarrow}{\rightarrow}
\newcommand{\tunion}{\bigcup}
\newcommand{\tall}{\forall}

\newcommand{\kproc}{\texttt{proc}}
\newcommand{\kref}{\texttt{ref}}
\newcommand{\kint}{\texttt{int}}

\newcommand{\torigin}{\lozenge}
\newcommand{\uorigin}{\blacklozenge}

\newcommand{\vref}[1]{\textbf{ref}~#1}
\newcommand{\vlam}[2]{\lambda #1.\, #2}
\newcommand{\efst}[1]{\mathsf{fst}\,#1}

\newcommand{\vrel}[2]{#1 \models #2}
\newcommand{\vstate}[2]{[#1~#2]}
\newcommand{\vstategen}{\vstate{\sigma}{v}}

\newcommand{\tfun}[5]{\mathbf{fun} #1 (#2 : #3) \rightarrow #4\,.\,#5}
\newcommand{\ufun}[3]{\mathbf{fun} #1 (#2)\,.\,#3}

%% value relation



%% -----------------------------------------------------------------------------

\newcommand{\mnowm}{\blacksquare}
\newcommand{\mnow}{\square}
\newcommand{\mlater}{\diamond}

%% END custom tex styles
%% =============================================================================
